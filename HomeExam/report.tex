\documentclass[11pt,twoside,swedish]{article}
\usepackage{epsfig}
\usepackage{graphicx}
\usepackage{subfigure}
\usepackage{amssymb}
\usepackage{latexsym}
\usepackage{epic}
\usepackage{eepic}
\usepackage{amsmath}
\usepackage{amsthm}
\usepackage{listings}
\usepackage[useregional]{datetime2}
\usepackage{fullpage}
\usepackage{wrapfig}
\usepackage{todonotes}
\usepackage{multirow,array}
\usepackage{subfiles}


\setlength{\parskip}{0cm plus0mm minus0mm}

\newcommand*{\mtodo}[1]{\todo[color=violet!50]{Mikael: #1}}
\newcommand{\Mtodo}[1]{\todo[inline,color=violet!50]{Mikael: #1}}
\usepackage{comment}

\usepackage[natbib=true,citestyle=authoryear,backend=bibtex,useprefix]{biblatex}

\addbibresource{library.bib}


\usepackage{tikz}
\usetikzlibrary{shapes,arrows}

%\newtheorem{theorem}{Theorem}[section]
%\newtheorem{lemma}[theorem]{Lemma}
%\newtheorem{proposition}[theorem]{Proposition}
%\newtheorem{definition}[theorem]{Definition}
%\newtheorem{ex}{Example}[chapter]
%\newtheorem{thm}{Theorem}[chapter]
%\newtheorem{lem}{Lemma}[chapter]
%\newtheorem{cor}{Corollary}[chapter]
%\newtheorem{defn}{Definition}[chapter]
%\newtheorem{form}{Form}[chapter]
%\newtheorem*{Proof}{\it Proof}
%\newcommand{\pbox}{\hspace*{1cm}\parbox{12cm}}


\newcommand{\bx}{\hspace*{\fill}$\square$}

\renewcommand{\labelitemi}{$\cdot$}

%\textwidth 146 mm
%\textheight 230 mm
%\oddsidemargin 7mm \evensidemargin -1mm \topmargin -4mm

\begin{document}
\title{MSA220, Statistical Learning for Big Data\\
  Final Spring 2018}
\author{Mikael B\"o\"ors}

\clearpage
\maketitle
\thispagestyle{empty}

\newpage
\clearpage
\tableofcontents
\thispagestyle{empty}
\newpage

\setcounter{page}{1}

\section{Question 1}\label{Question 1}

\subsection{Mini 1}\label{Mini 1}

In order to evaluate the result of a clustering algorithm, many heuristics and
rules of thumb have been suggested. In this mini analysis I and my
group compared a few of these heuristics and evaluated their
performance on different clustering algorithms and data sets.

We chose to use artificial data sets for this task. The motivation for
this choice was that in order to evaluate the performance of the
clustering evaluation methods, we needed to know how many clusters the
data actually contained and this can be controlled when generating
artificial data. Also, the function we used to generate data allowed
us to control the shapes of the clusters, their separation, the number
of outliers and similar cluster properties. Our simulated data sets
were designed to consist of four clusters with varying spread and
numbers of observations. The generated data sets were two-dimensional
since we wanted to be able to visualize the clusters and the results
of the clustering algorithms. Also, to keep the analysis as clean as
possible we did not include any outliers or similar impurities.

The clustering algorithms we used were the EM-algorithm, K-means and
PAM (K-medoids). K-means and PAM are very similar algorithms that both
tries to choose n points in the data space that minimizes the sum of the
distances to the data points closest to them. The difference between
them is that K-means can use an arbitrary point as the cluster center,
while PAM uses observations as the medoids. Unlike K-means and PAM,
the EM algorithm is an model based clustering algorithm that tries to
find model parameters for multivariate normal distributions that
maximizes the posterior (log-) likelihood of the model. We chose to
include a model based algorithm in order to evaluate the clustering
evaluation methods on different types of algorithms.

The clustering evaluation methods we used were the average silhouette
score and the Dunn index. Both of these methods that can be applied to
the result of any clustering algorithm since they do not use the model
itself to evaluate the performance. The silhouette score for an
observation is a score that uses the average distance to all other
observations in the same cluster and the minimal average distance to
all other observations in another cluster to score how similar the
observation is to the observations in the same cluster relative to how
similar it is to the points in another cluster. The average silhouette
score is the average silhouette score for all observations. A high
average silhouette score indicates that the clusters are dense and
well separated. One can use this to determine how many clusters to
look for by choosing the number of clusters that maximizes the average
silhouette score. Similarly, the Dunn index is the ratio between the
minimal distance between observations in different clusters and the maximum
distance between observations in the same cluster. A large Dunn index
indicates dense well separated clusters and hence one can use the
index to choose the number of clusters that yields the largest Dunn
index. Since the EM algorithm is a model based algorithm, one can use
the Bayesian information criteria (BIC) to evaluate the model. The BIC
score measures how well the model fits the data while also punishing
complex models (i.e.\ it prefers simple models). A method to choose
the number of clusters is to choose the number of clusters that
minimizes the BIC score.

The experiment setup was to generate data sets containing four
clusters with different amounts of separation (-0.5, -0.4,...,0.5),
apply the three clustering algorithms with the parameter "number of
clusters" equal to 2,3,...,10 and register what number of clusters the
different evaluation methods chose. This was repeated 100 times and
the results were visualized as histograms. To provide an idea of how
the data sets looked with different amounts of separation, examples of
the generated data are visualized in Figure~\ref{facit}. Negative
separation value means that the intended clusters are overlapping
while a strictly positive separation value means that the intended
clusters are well separated. Also note that even though the clusters
have different spread and consist of different numbers of
observations, they all have more or less the same shape (they are all
more or less spherical clusters).

\graphicspath{{/home/mikael/Repos/Courses/msa220/Mini/Mini/Mini_1/}}
\begin{figure}[!h]
\begin{center}
\includegraphics[scale=0.5]{Facit.png}
\caption{Examples of generated clusters with different amounts of separation.}
\label{facit}
\end{center}
\end{figure}

The results of the experiments are presented as histograms in
Appendix~\ref{Plots Mini 1}. The histograms show that for all of the
clustering algorithms, all evaluation methods failed in finding the
correct number of clusters when the separation value was negative,
i.e.\ when the clusters were overlapping. This is not very
surprising. Both K-means and PAM will have difficulties to converge to
good cluster centers when the clusters are not well separated and the
same is true for the EM algorithm. Therefore the bad result might be
partly due to the failure of the clustering algorithms rather than the
evaluation methods. However, neither the average silhouette score
method or the Dunn index methods are designed to be used when the
clusters are not well separated. When clusters are overlapping, the
silhouette score will be small since the difference between the
average within cluster distances and the minimal average distance
between groups will be small. This makes it hard for the method to
differ between a good and a bad clustering. We also expect the method
to be a bit conservative when choosing the number of clusters, and
this was confirmed in this experiment since in most of the cases with
negative separation it chose fewer than 4 clusters. The Dunn index
method suffers from similar problems. When clusters are overlapping,
the Dunn ratio will be small even for a good clustering since the
minimal between cluster distance is smaller than the maximum within
cluster distance. Also, unlike the silhouette method, the Dunn method
seems to prefer a larger number of clusters when the clusters are not
well separated. This is not a surprise since when using many clusters,
the maximal within cluster distance will be smaller while the minimal
between clusters distance will remain about the same. The BIC
evaluation method for the EM algorithm chose fewer than 4 clusters
when the clusters were overlapping. Again, this is not a surprise
since the gain in likelihood obtained with more clusters will be small
compared to the increase in complexity.

In the cases where the clusters were well separated, all combinations
of evaluation method and clustering algorithm performed
well. K-means in combination with the silhouette method and the Dunn
method performed slightly worse than the other combinations. I think
this was due to poor performance of the K-means algorithm rather than
the evaluation methods. When I visualized the clustering, K-means had
a tendency to split a real cluster into two, i.e.\ one ore more of
the cluster means did not converge to a true cluster mean but rather
converged to a point between two clusters. Since PAM uses medoids
rather than mean points, it is more stable than K-means and did not
suffer from this problem, which explains the difference in the result
(at least in part). Also, the method used for generating clusters
compliments the EM algorithm since the clusters are normally
distributed and the EM algorithm assumes that the clusters comes from
a collection of normal distributions. This might explain why the combination
of the EM algorithm and the BIC method performed good when the
separation value was zero while the other methods failed. Even though
the EM algorithm most likely performed good when the number of
clusters was set to four and the separation was zero, the silhouette
method and the Dunn method were handicapped by the lack of separation,
while BIC managed to detect the good clustering sine it uses
likelihood rather than separation and denseness to evaluate the
result. This brings me to the flaws of the experiment setup.

The generated clusters were all ruffly spherical and both K-means and PAM are
designed to find these types of clusters. This makes the evaluation of
the evaluation methods a bit biased, since it is likely that the
result would have looked different if the clusters would have had
other shapes. Also, both the silhouette method the Dunn method were
designed to perform good on spherical clusters. Again, this adds bias
to the results. Another problem is the one mentioned above, namely
that the clusters were normally distributed and therefore fits the
model assumption of the EM algorithm perfectly. If one were to extend
this experiment, it would probably be a good idea to use data that is
not designed to fit the models and to use more diverge clustering
algorithms since both K-means and PAM are very similar. Even though
the experiment setup was biased, I still think that the  results
illustrates the difficulty in finding the correct number of clusters.

\subsection{Mini 2}\label{Mini 2}
In this Mini analysis I compared the performance of different
classifiers and I also investigated two ensemble methods (bagging and
stacking) in order to see if any of them were able to improve on the
result obtained from using single classifiers. The original task
stated that we were supposed to compare the classifiers and ensembles
on 2-3 different data sets. However, I was working alone on this mini
and due to time limitations I was only able to analyse the result for
one data set since I put a lot of effort into feature engineering in
and pre-processing of the data. I hope that this deviation from the
instructions is acceptable.

The data I chose for this project was the Titanic data set from
Kaggle. It consists of 891 observations were each observation contains
information about a passenger on the Titanic. It has 12 columns, one
of which is a binary column (Survived) that states if the passenger
survived (1) or not (0). I chose this column to be my target column,
i.e.\ this is a binary classification task. The remaining 11 columns
contains information such as the sex of the passenger, his or her
passenger class (1,2,3) and so on. Some of the categorical columns
contained nearly as many levels as there were observations and could
not be used without some sort of pre-processing. For example, the
"Name" column contained the title and full name of the passenger. I
chose to extract the title from this column and drop the name of the
passenger since there were only a few titles in the data but every
name was unique. I treated some of the other categorical features
similarly (by for example extracting the ship section from the "Cabin"
column and discarding the unique cabin number), and an overview of all
the columns after pre-processing is presented in
Table~\ref{titanic features}. The "Age" column was numerical, but it
had missing values. I chose to impute the missing values with 0 and
add a dummy variable to indicate if the value was missing or not. In
retrospect, this was probably not the best thing to do since the
imputed values skewed the age-distribution. It would probably have
been better to either impute the median age or to use regression on
the other variables to impute the age.

\begin{table}
  \begin{center}
\begin{tabular}{ l l }
 Feature & Type \\ 
 \hline
 Pclass &   Numeric: 1,2,3\\
Title & Categories: Mr, Master, Mrs, Miss, Other\\
Sex & Categories: Male, Female\\
Age & Numeric\\
SibSp & Numeric: Number of siblings/spouses\\
Parch & Numeric: Number of parents/children\\
Fare & Numeric: Ticket price\\
Cabin & Categories: n, B, C, E, D, F, G, A\\
Embarked & Categories: C, S, Q\\
Age Missing & Dummy
\end{tabular}
\end{center}
\label{titanic features}
\end{table}

I used one hot encoding to represent the categorical features when
applying the various classifiers. For most of the classifiers, this is
a viable representation of categorical features. However, it is not a
good option when using logistic regression since the dummy
coefficients will be linearly dependent and thus cannot be used for
analysis. I did not think of this problem and if I would have tried to
interpret which variables that were significant for the probability of
survival it is likely that I would come to the wrong
conclusion. However, since I only used the model for prediction my
mistake should not affect the results.

Before applying any classifiers I split the data into a train- and a
test set. Since the original data set was small I did not want to
sacrifice to much data for testing and therefore I used a 90/10
split, resulting in 802 observations in the training data and 89
observations in the test data. I would have preferred to divide the
data into a train-, a validation- and a test set and use the
validation set to tune the hyper parameters in the different models.
But because of the small data set I instead chose to use 10-fold cross
validation for this purpose. The classifiers I compared in
this project were logistic regression, support vector classifiers
(SVC), CART, random forest, k-nearest neighbour classifiers (KNN), gradient
boosted classifiers (GBC), linear discriminant analysis (LDA), Naive
Bayes (NB), a neural network with three hidden layers (NN) and
quadratic discriminant analysis (QDA).

The results of the classifiers
are presented in Table~\ref{titanic classifiers result}. The results
were a bit surprising since there are three groups of classifiers that
got the exact same accuracy within the groups and there does not seem
to be any obvious similarities between the classifiers in the
groups. This rises the suspicion that many of the classifiers made the
exact same mistakes. Indeed, when I analyzed the miss-classified
observations for the different classifiers, most of them were
identical. What is more the set of observation that were
miss-classified by LR and SVC were a subset of the set of observations
that were miss-classified by most of the other classifiers. The fact
that LR and SVC (with a linear kernel) performed best indicates that there was a blurry
linear separation between the classes since both of these methods use
linear decision boundaries. I therefore suspect that the classes were
almost linearly separated, meaning that the separation was not perfect
but had some overlap. If this was the case it could explain why CART
and RF performed slightly worse since they would try to estimate the
linear "separation" using step functions. The same would be true for
GBC which, similarly to RF, uses several decision trees. LDA uses a
linear decision boundary as well. However it has in general more
trouble with finding the optimal boundary when the classes are not
perfectly separated than methods such as LR and SVC. NB and QDA uses
non-linear decision boundaries and both of these performed relatively
bad. However QDA requires more data than LDA and it is possible that
it was impaired by the small data set in this case. NN generally
requires a lot of training data as well in order to find a good
decision boundary and it is also possible that the architecture I used
was not suitable for the problem.

\begin{table}
\begin{center}
\begin{tabular}{ l | l l l }
 Classifier & Accuracy & P/R Survived & P/R Died \\ 
 \hline
 \hline
 LR & 0.876 & 0.86/0.91 & 0.90/0.83\\
SVC & 0.876 & 0.86/0.91 & 0.90/0.83\\
\hline
 CART & 0.865 & 0.84/0.91 & 0.89/0.81\\
RF & 0.865 & 0.84/0.91 & 0.89/0.81\\
KNN & 0.865 & 0.83/0.94 & 0.92/0.79\\
GBC & 0.865 & 0.82/0.96 & 0.94/0.76\\
LDA & 0.865 & 0.84/0.91 & 0.89/0.81\\
\hline
NB & 0.854 & 0.90/0.81 & 0.81/0.90\\
NN & 0.843 & 0.81/0.91 & 0.89/0.76\\
QDA & 0.820 & 0.82/0.85 & 0.82/0.79\\
\end{tabular}
\end{center}
\label{titanic classifiers result}
\end{table}

In Table~\ref{titanic classifiers result} the precision and recall for
the two classes are presented for all of the classifiers. Most of the
classifiers seems to have less precision when predicting survivors
than non-survivors while they have higher recall for the survivors
than the non-survivors. This suggests that the non-survivor class had
more spread than the survivor class. This is not very surprising since
some of the features levels, such as Sex: Female and PClass: 1,2, are
strong indicators of survival as can be seen in Figure~\ref{titanic
  sex pclass} in Appendix~\ref{Plots Mini 2}. 

I also tried stacking and bagging. Since the two best classifiers made
the exact same mistakes and since the mistakes done by these
classifiers were also made by the other classifiers, I did not expect
staking several classifiers to improve the accuracy at all. I was
right. I tried several different combinations of classifiers and they
all ended up with the same accuracy as one of the single classifiers
or a lower accuracy than the best classifiers in the
ensemble. The idea with stacking is that the errors or biases of
different classifiers are canceled out when using average
voting. However if all classifiers in the ensemble makes the same
mistakes there is nothing to gain from combining them.

I had higher hopes for bagging, since it is possible that using
slightly different training data for several classifiers of the same
type could improve the result. However the only classifier that
improved using bagging was CART as can be seen in Table~\ref{titanic
  bagging}. All other classifiers got the exact same accuracy when
bagged except for KNN which got a lower accuracy using bagging. This
suggests that the classifiers were fairly stable and not very
sensitive to small changes in the training data. KNN is sensitive to
small changes in the training data when using a small neighborhood (as
I did) and it is possible that the relatively good result obtained
from using a single KNN was due to luck. However I did not expect CART
to improve when bagged in this case, since CART and RF got the same
result when using a single classifier. I had expected this CART to be
stable, but it is clear that bagging was able to improve the result to
that obtained from a single LR and SVC. If my suspicion that the
classes were linearly separable except for some noise were correct, it is possible
that the decision boundary of the CART ensemble was a finer step
function that better approximated the linear separation.

\begin{table}
\begin{center}
\begin{tabular}{ l | l l}
 Classifier & Accuracy without bagging & Accuracy with bagging \\ 
 \hline
 CART & 0.865 & 0.876 \\
 KNN & 0.865 & 0.854\\
\end{tabular}
\end{center}
\label{titanic bagging}
\end{table}



\subsection{Mini 3}\label{Mini 3}
\subsection{Mini 4}\label{Mini 4}
\subsection{Mini 5}\label{Mini 5}

\section{Question 2}\label{Question 2}

\section{Question 3}\label{Question 3}

\newpage
\appendix

\section{Plots Mini 1}\label{Plots Mini 1}

\graphicspath{{/home/mikael/Repos/Courses/msa220/Mini/Mini/Mini_1/new_plots/}}
\begin{figure}
\begin{center}
\includegraphics[scale=0.5]{em_sil.png}
\caption{Histograms showing the optimal number of clusters for the EM
  algorithm according to the silhouette score method. The different
  histograms shows the results for artificially generated data with 4
  intended clusters and different separation coefficients.}
\label{em_sil_hists}
\end{center}
\end{figure}

\begin{figure}
\begin{center}
\includegraphics[scale=0.5]{em_dunn.png}
\caption{Histograms showing the optimal number of clusters for the EM
  algorithm according to the Dunn index method. The different
  histograms shows the results for artificially generated data with 4
  intended clusters and different separation coefficients.}
\label{em_dunn_hists}
\end{center}
\end{figure}

\begin{figure}
\begin{center}
\includegraphics[scale=0.5]{em_BIC.png}
\caption{Histograms showing the optimal number of clusters for the EM
  algorithm according to the BIC method. The different
  histograms shows the results for artificially generated data with 4
  intended clusters and different separation coefficients.}
\label{em_BIC_hists}
\end{center}
\end{figure}

\begin{figure}
\begin{center}
\includegraphics[scale=0.5]{km_sil.png}
\caption{Histograms showing the optimal number of clusters for the k-means
  algorithm according to the silhouette score method. The different
  histograms shows the results for artificially generated data with 4
  intended clusters and different separation coefficients.}
\label{km_sil_hists}
\end{center}
\end{figure}

\begin{figure}
\begin{center}
\includegraphics[scale=0.5]{km_dunn.png}
\caption{Histograms showing the optimal number of clusters for the k-means
  algorithm according to the Dunn index method. The different
  histograms shows the results for artificially generated data with 4
  intended clusters and different separation coefficients.}
\label{km_dunn_hists}
\end{center}
\end{figure}


\begin{figure}
\begin{center}
\includegraphics[scale=0.5]{pam_sil.png}
\caption{Histograms showing the optimal number of clusters for the PAM
  algorithm according to the silhouette score method. The different
  histograms shows the results for artificially generated data with 4
  intended clusters and different separation coefficients.}
\label{pam_sil_hists}
\end{center}
\end{figure}

\begin{figure}
\begin{center}
\includegraphics[scale=0.5]{pam_dunn.png}
\caption{Histograms showing the optimal number of clusters for the PAM
  algorithm according to the Dunn index method. The different
  histograms shows the results for artificially generated data with 4
  intended clusters and different separation coefficients.}
\label{pam_dunn_hists}
\end{center}
\end{figure}


\newpage
\section{Plots Mini 2}\label{Plots Mini 2}

\graphicspath{{/home/mikael/Repos/Courses/msa220/Mini/Mini2/Titanic/}}
\begin{figure}
\begin{center}
\includegraphics[scale=0.5]{sex_class.png}
\caption{The survival proportions for different sex/passenger class combinations.}
\label{titanic sex pclass}
\end{center}
\end{figure}


\newpage
\printbibliography



\end{document}
